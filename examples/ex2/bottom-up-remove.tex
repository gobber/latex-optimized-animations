\documentclass[10pt]{article}

\usepackage[brazil]{babel} 
\usepackage[utf8]{inputenc}

\usepackage[T1]{fontenc}
\usepackage{lmodern}
\usepackage{pgfplots}
\usepackage{intcalc}
\usepackage{amsmath}
\usepackage{amsfonts}
\usepackage{comment}
\usepackage{multirow}		
\usepackage{arydshln} 		
\usepackage{amssymb}
\usepackage{tikz,tikz-3dplot} 
\usepackage{arrayjobx}	
\usepackage{trimspaces}	
\usepackage{xifthen}
\usepackage[tightpage,active]{preview}

\usetikzlibrary{backgrounds}
\usetikzlibrary{mindmap,trees,positioning,calc}	

\newcommand{\volume}{\mathop{\ooalign{\hfil$V$\hfil\cr\kern0.08em--\hfil\cr}}\nolimits}

	\tikzstyle{hasseNode}=[draw, circle, inner sep=2pt, outer sep=0pt, node distance=0cm]	
	
	\tikzstyle{pixelBother}=[green!30]	
	
	\makeatletter
		\def\trimspace#1{\trim@spaces@in{#1}}
	\makeatother

	\newcommand{\drawBinaryImage}[3]{%
	   
	    \pgfmathtruncatemacro{\gridwidth}{#1}
	    \pgfmathtruncatemacro{\gridheight}{#2}
	    \dataheight=\gridwidth
	    
	    \draw[gray,densely dashed] (0,0) grid (\gridwidth,\gridheight);
	    
	    \foreach \x in {1,...,\gridwidth}{% 
	    	\foreach \y in {1,...,\gridheight}{%
	    	   
	            \pgfmathtruncatemacro{\colnum}{\x}
	            \pgfmathtruncatemacro{\rownum}{\gridheight+1-\y}
	            \expandafter\csname check#3\endcsname(\rownum,\colnum)\trimspace\cachedata
	            
	            \ifthenelse{\cachedata=0}
	            	{\fill[black,fill opacity=1] (\x-0.9,\y-0.9) rectangle (\x-0.1,\y-0.1);}
	            	{}
	            	
				\ifthenelse{\cachedata=1}
	            	{\fill[white,fill opacity=1] (\x-0.9,\y-0.9) rectangle (\x-0.1,\y-0.1);}
	            	{}
	            	
				\ifthenelse{\cachedata=2}
	            	{\fill[red,fill opacity=1] (\x-0.9,\y-0.9) rectangle (\x-0.1,\y-0.1);}
	            	{}	       
	            	
	            % for other purposes
	            	
				\ifthenelse{\cachedata=3}
	            	{\fill[pixelBother,fill opacity=1] (\x-0.9,\y-0.9) rectangle (\x-0.1,\y-0.1);}
	            	{}	
	            
	     	}
	    }
	    
	}
	
	% draw a gray image
	
	\tikzstyle{textTreeLight}=[font=\fontsize{4pt}{2pt}\selectfont, text=white]

	\tikzstyle{textTreeDark}=[font=\fontsize{4pt}{2pt}\selectfont, text=black!80]
	
	\tikzstyle{textTreeOther}=[font=\fontsize{4pt}{2pt}\selectfont, text=orange]
	
	\tikzstyle{pixelTreeOther}=[green!30]
	
	\newcommand{\drawGrayImage}[3]{%
	   
		\pgfmathtruncatemacro{\gridwidth}{#1}
		\pgfmathtruncatemacro{\gridheight}{#2}
		\dataheight=\gridwidth
	
		\draw[gray,densely dashed] (0,0) grid (\gridwidth,\gridheight);
		\foreach \x in {1,...,\gridwidth}{% 
			\foreach \y in {1,...,\gridheight}{%
			   
				\pgfmathtruncatemacro{\colnum}{\x}
				\pgfmathtruncatemacro{\rownum}{\gridheight+1-\y}
				\expandafter\csname check#3\endcsname(\rownum,\colnum)\trimspace\cachedata
			
				\ifthenelse{\cachedata=0}
				{
					\fill[black,fill opacity=1] (\x-0.9,\y-0.9) rectangle (\x-0.1,\y-0.1);        
					\draw[textTreeLight, fill opacity=1] (\x-0.5,\y-0.5)  node[] {$\cachedata$}; 
				}
				{}
			
				\ifthenelse{\cachedata=1}
				{
					\fill[black,fill opacity=0.85] (\x-0.9,\y-0.9) rectangle (\x-0.1,\y-0.1);        
					\draw[textTreeLight, fill opacity=1] (\x-0.5,\y-0.5)  node[] {$\cachedata$};
				}
				{}
			
				\ifthenelse{\cachedata=2}
				{
					\fill[black,fill opacity=0.7] (\x-0.9,\y-0.9) rectangle (\x-0.1,\y-0.1);        
					\draw[textTreeLight, fill opacity=1] (\x-0.5,\y-0.5)  node[] {$\cachedata$};
				}
				{}
			
				\ifthenelse{\cachedata=3}
				{
					\fill[black,fill opacity=0.55] (\x-0.9,\y-0.9) rectangle (\x-0.1,\y-0.1);        
					\draw[textTreeLight, fill opacity=1] (\x-0.5,\y-0.5)  node[] {$\cachedata$};
				}
				{}
			
				\ifthenelse{\cachedata=4}
				{
					\fill[black,fill opacity=0.25] (\x-0.9,\y-0.9) rectangle (\x-0.1,\y-0.1);     
					\draw[textTreeDark, fill opacity=1] (\x-0.5,\y-0.5)  node[] {$\cachedata$};   
				}
				{}
			
				\ifthenelse{\cachedata=5}
				{
					\fill[black,fill opacity=0.2] (\x-0.9,\y-0.9) rectangle (\x-0.1,\y-0.1);        
					\draw[textTreeDark, fill opacity=1] (\x-0.5,\y-0.5)  node[] {$\cachedata$};   
				}
				{}
			
				\ifthenelse{\cachedata=6}
				{
					\fill[black,fill opacity=0.1] (\x-0.9,\y-0.9) rectangle (\x-0.1,\y-0.1);        
					\draw[textTreeDark, fill opacity=1] (\x-0.5,\y-0.5)  node[] {$\cachedata$};   
				}
				{}
			
				\ifthenelse{\cachedata=7}
				{
					\fill[black,fill opacity=0] (\x-0.9,\y-0.9) rectangle (\x-0.1,\y-0.1);        
					\draw[textTreeDark, fill opacity=1] (\x-0.5,\y-0.5)  node[] {$\cachedata$};   
				}
				{}				
				
				% for other purposes

				\ifthenelse{\cachedata=90}
				{
					\fill[pixelTreeOther] (\x-0.9,\y-0.9) rectangle (\x-0.1,\y-0.1);
					\draw[textTreeOther] (\x-0.5,\y-0.5)  node[] {$0$}; 					
				}
				{}				

				\ifthenelse{\cachedata=91}
				{
					\fill[pixelTreeOther] (\x-0.9,\y-0.9) rectangle (\x-0.1,\y-0.1);
					\draw[textTreeOther] (\x-0.5,\y-0.5)  node[] {$1$}; 					
				}
				{}				

				\ifthenelse{\cachedata=93}
				{
					\fill[pixelTreeOther] (\x-0.9,\y-0.9) rectangle (\x-0.1,\y-0.1);
					\draw[textTreeOther] (\x-0.5,\y-0.5)  node[] {$3$}; 					
				}
				{}				
				
				\ifthenelse{\cachedata=95}
				{
					\fill[pixelTreeOther] (\x-0.9,\y-0.9) rectangle (\x-0.1,\y-0.1);
					\draw[textTreeOther] (\x-0.5,\y-0.5)  node[] {$5$}; 					
				}
				{}								
				
				\ifthenelse{\cachedata=97}
				{
					\fill[pixelTreeOther] (\x-0.9,\y-0.9) rectangle (\x-0.1,\y-0.1);
					\draw[textTreeOther] (\x-0.5,\y-0.5)  node[] {$7$}; 					
				}
				{}
															
			}
		}
	}	
		
%

\begin{document}

\tikzstyle{hasseNode}=[draw, yshift=2cm, circle, inner sep=2pt, outer sep=0pt, node distance=0cm]	

	\newarray\treeRoot
	\readarray{treeRoot}{%
		2 & 2 & 2 & 2 & 2 & 2 & 2 & 2 &
		2 & 0 & 0 & 0 & 0 & 0 & 0 & 2 &
		2 & 0 & 0 & 0 & 0 & 0 & 0 & 2 &
		2 & 0 & 0 & 0 & 0 & 0 & 0 & 2 &
		2 & 0 & 0 & 0 & 0 & 0 & 0 & 2 &
		2 & 0 & 0 & 0 & 0 & 0 & 0 & 2 &
		2 & 0 & 0 & 0 & 0 & 0 & 0 & 2 &
		2 & 2 & 2 & 2 & 2 & 2 & 2 & 2}

	\newarray\treeNodeOne
	\readarray{treeNodeOne}{%
		1 & 1 & 1 & 1 & 1 & 1 & 1 & 1 &
		1 & 2 & 2 & 2 & 2 & 2 & 2 & 1 &
		1 & 2 & 0 & 0 & 0 & 0 & 2 & 1 &
		1 & 2 & 0 & 0 & 0 & 0 & 2 & 1 &
		1 & 2 & 0 & 0 & 0 & 0 & 2 & 1 &
		1 & 2 & 0 & 0 & 0 & 0 & 2 & 1 &
		1 & 2 & 2 & 2 & 2 & 2 & 2 & 1 &
		1 & 1 & 1 & 1 & 1 & 1 & 1 & 1}

	\newarray\treeNodeTwo
	\readarray{treeNodeTwo}{%
		1 & 1 & 1 & 1 & 1 & 1 & 1 & 1 &
		1 & 1 & 1 & 1 & 1 & 1 & 1 & 1 &
		1 & 1 & 2 & 2 & 2 & 2 & 1 & 1 &
		1 & 1 & 2 & 0 & 0 & 2 & 1 & 1 &
		1 & 1 & 2 & 0 & 0 & 2 & 1 & 1 &
		1 & 1 & 2 & 2 & 2 & 2 & 1 & 1 &
		1 & 1 & 1 & 1 & 1 & 1 & 1 & 1 &
		1 & 1 & 1 & 1 & 1 & 1 & 1 & 1}

	\newarray\treeNodeThree
	\readarray{treeNodeThree}{%
		1 & 1 & 1 & 1 & 1 & 1 & 1 & 1 &
		1 & 1 & 1 & 1 & 1 & 1 & 1 & 1 &
		1 & 1 & 1 & 1 & 1 & 1 & 1 & 1 &
		1 & 1 & 1 & 2 & 2 & 1 & 1 & 1 &
		1 & 1 & 1 & 2 & 2 & 1 & 1 & 1 &
		1 & 1 & 1 & 1 & 1 & 1 & 1 & 1 &
		1 & 1 & 1 & 1 & 1 & 1 & 1 & 1 &
		1 & 1 & 1 & 1 & 1 & 1 & 1 & 1}

	\newarray\treeNodeOneMerge
	\readarray{treeNodeOneMerge}{%
		1 & 1 & 1 & 1 & 1 & 1 & 1 & 1 &
		1 & 2 & 2 & 2 & 2 & 2 & 2 & 1 &
		1 & 2 & 2 & 2 & 2 & 2 & 2 & 1 &
		1 & 2 & 2 & 0 & 0 & 2 & 2 & 1 &
		1 & 2 & 2 & 0 & 0 & 2 & 2 & 1 &
		1 & 2 & 2 & 2 & 2 & 2 & 2 & 1 &
		1 & 2 & 2 & 2 & 2 & 2 & 2 & 1 &
		1 & 1 & 1 & 1 & 1 & 1 & 1 & 1}

	\newsavebox\treeRootCashTwo
	\begin{lrbox}{\treeRootCashTwo}
		\begin{tikzpicture}[scale=0.15]
			\drawBinaryImage{8}{8}{treeRoot}
		\end{tikzpicture}
	\end{lrbox}

	\newsavebox\treeNodeOneCashTwo
	\begin{lrbox}{\treeNodeOneCashTwo}
		\begin{tikzpicture}[scale=0.15]
			\drawBinaryImage{8}{8}{treeNodeOne}
		\end{tikzpicture}
	\end{lrbox}
	
	\newsavebox\treeNodeTwoCashTwo
	\begin{lrbox}{\treeNodeTwoCashTwo}
		\begin{tikzpicture}[scale=0.15]
			\drawBinaryImage{8}{8}{treeNodeTwo}
		\end{tikzpicture}
	\end{lrbox}

	\newsavebox\treeNodeThreeCashTwo
	\begin{lrbox}{\treeNodeThreeCashTwo}
		\begin{tikzpicture}[scale=0.15]
			\drawBinaryImage{8}{8}{treeNodeThree}
		\end{tikzpicture}
	\end{lrbox}

	\newsavebox\treeNodeOneMergeCash
	\begin{lrbox}{\treeNodeOneMergeCash}
		\begin{tikzpicture}[scale=0.15]
			\drawBinaryImage{8}{8}{treeNodeOneMerge}
		\end{tikzpicture}
	\end{lrbox}
	
	\def\N{4}
			
	\def\M{101}
		 	
	\tikzset{label position/.style={
		 	    right= 0cm of #1, xshift=-0.3cm, yshift=0.7cm, font=\scriptsize
	}}
			
	\tikzset{parameters position/.style={
		 	    below= 0.05cm of #1, font=\tiny, align=center, text width=2cm
	}}
						
	\tikzstyle{equation}=[font=\scriptsize, align=center, yshift=-1.1cm]
			
	% % %
			
	\tikzset{root shift/.style={
				xshift= #1cm
	}}
			
	% var global definitions %
								
	\xdef\rootXShift{-4}
			
	\xdef\nOneXShift{-1.5}
			
	\xdef\nTwoXShift{1}
				
	\xdef\nThreeXShift{3.5}

	\xdef\blankcolorRoot{black}			
			
	\xdef\blankcolorNodeOne{black}			

	\xdef\blankcolorNodeTwo{black}	

	\xdef\blankcolorNodeThree{black}
			
	\xdef\boxNodeOne{\treeNodeOneCashTwo}
			
	\xdef\volRoot{140}
			
	\xdef\areaRoot{28}
			
	\xdef\contourRoot{32}	
			
	\xdef\volNodeOne{60}
			
	\xdef\areaNodeOne{20}
			
	\xdef\contourNodeOne{24}		

	\xdef\volNodeTwo{12}
			
	\xdef\areaNodeTwo{12}
			
	\xdef\contourNodeTwo{16}
			
	\xdef\volNodeThree{28}
			
	\xdef\areaNodeThree{4}
			
	\xdef\contourNodeThree{8}		
											
	%
			
	\xdef\nodeFatherVol{\volNodeTwo}			
			
	\xdef\nodeChildVol{\volNodeThree}
			
	\xdef\nodeUnionVol{40}

	\xdef\nodeFatherArea{\areaNodeTwo}			
			
	\xdef\nodeChildArea{\areaNodeThree}
		
	\xdef\nodeUnionArea{16}
						
	\xdef\nodeContour{\contourNodeThree}
			
	\xdef\blankcolorFather{black}
			
	\xdef\blankcolorChild{black}
			
	%
			
	\xdef\nodeComputing{3}
			
	\xdef\result{-92}
			
	\xdef\signal{<}	
	
	\PreviewEnvironment{tikzpicture}
		
	\xdef\i{1}
	\whiledo{\i<\N}{%
	\begin{tikzpicture}	
		\useasboundingbox (-5.5 , -1.7) 
							rectangle 
							(5 , 3);			
					
					\node[hasseNode, xshift=-4cm](root){\usebox\treeRootCashTwo};
					\node[label position=root]{$ \tau_0 $,{\color{red}$ \hat{\tau_0} $}};
					
					% tree %	
					
					\node[left = 0.05cm of root]{\scriptsize $ \mathcal{T}_i $};
					
			  		\node[hasseNode, xshift=-1.5cm] (n1) {\usebox\treeNodeOneCashTwo};
					\node[label position= n1]{$ \tau_1 $,{\color{red}$ \hat{\tau_1} $}};				  						
													  							  						
				  	\node[hasseNode, xshift = 1cm] (n2) {\usebox\treeNodeTwoCashTwo};
				  	\node[hasseNode, xshift = 3.5cm] (n3) {\usebox\treeNodeThreeCashTwo};
				  		
				  	\node[label position= n2]{$ \tau_2 $,{\color{red}$ \hat{\tau_2} $}};
				  	\node[label position= n3]{$ \tau_3 $,{\color{red}$ \hat{\tau_2} $}};
									  			
			  		\draw[] (root) -- (n1)
					  		(n1) -- (n2)
					  		(n2) -- (n3);

			  		\ifthenelse{\i>1}
			  			{
							\draw[green!60, -latex] ($(n3.west)-(0,0.1)$) -- ($(n2.east)-(0,0.1)$);
			  			}
			  			{}
			  			
			  		\ifthenelse{\i>2}
			  			{
							\draw[green!60, -latex] ($(n2.west)-(0,0.1)$) -- ($(n1.east)-(0,0.1)$);
			  			}
			  			{}

			  		\ifthenelse{\i>3}
			  			{
							\draw[green!60, -latex] ($(n1.west)-(0,0.1)$) -- ($(root.east)-(0,0.1)$);
			  			}
			  			{}
			  	
			  		\node[parameters position=root]{\baselineskip=7pt\vspace{-15pt}
			  		\begin{align*}\volume(\hat{\tau_0})&=140 \\ |\hat{\tau_0}|&=28 \\ |\partial \hat{\tau_0}|&=32 \end{align*} \par};
			  		
			  		\node[parameters position=n1]{\baselineskip=7pt\vspace{-15pt}
			  		\begin{align*}\volume(\hat{\tau_1})&=60 \\ |\hat{\tau_1}|&=20 \\ |\partial \hat{\tau_1}|&=24 \end{align*} \par};

			  		\node[parameters position=n2]{\baselineskip=7pt\vspace{-15pt}
			  		\begin{align*}\volume(\hat{\tau_2})&=12 \\ |\hat{\tau_2}|&=12 \\ |\partial \hat{\tau_2}|&=16 \end{align*} \par};

			  		\node[parameters position=n3]{\baselineskip=7pt\vspace{-15pt}
			  		\begin{align*}\volume(\hat{\tau_3})&=28 \\ |\hat{\tau_3}|&=4 \\ |\partial \hat{\tau_3}|&=8 \end{align*} \par};
			  		
			  		\node[equation]{$ \Delta E(\mathcal{T}_i, \tau)=\dfrac{[\volume(\hat{\tau}'')]^2}{|\hat{\tau}''|} - \dfrac{[\volume(\hat{\tau})]^2}{|\hat{\tau}|} - \dfrac{[\volume(\hat{\tau}')]^2}{|\hat{\tau}'|} + \nu \times ( |\partial \hat{\tau}| + |\partial \hat{\tau}'| - |\partial \hat{\tau}''| ) $};

	\end{tikzpicture}
	  \pgfmathtruncatemacro{\temp}{\i+1}%
	  \xdef\i{\temp}%
	  %\end{center}%
	  \newpage%
	}
	
	\xdef\i{1}
	\whiledo{\i<\M}{%
	\begin{tikzpicture}	
		\useasboundingbox (-5.5 , -1.7) 
							rectangle 
							(5 , 3);
			  		
			  		\pgfmathtruncatemacro{\div}{\intcalcMod{\i}{2}}
					
					\ifthenelse{\i>22 \AND \i<47}
						{
			  				% compute between node 1 and 2
			  				
							\xdef\nodeUnionVol{72}
							\xdef\nodeUnionArea{32}							
							\xdef\nodeContour{\contourNodeTwo}		
							
							\xdef\nodeFatherArea{\areaNodeOne}						
							\xdef\nodeFatherVol{\volNodeOne}							
							
							\xdef\nodeChildArea{\areaNodeTwo}						
							\xdef\nodeChildVol{\volNodeTwo}
								
							\xdef\nodeComputing{2}							
							\xdef\result{2}	
							\xdef\signal{>}	
							
						}
						{}
						
					\ifthenelse{\i>46} % merge nodes
						{
			  				\xdef\volNodeOne{72}
			  				\xdef\areaNodeOne{32}
			  				\xdef\boxNodeOne{\treeNodeOneMergeCash}						
						}
						{}

					\ifthenelse{\i>51 \AND \i<76}
						{
			  				% compute between node 1 and node 3
			  				
							\xdef\nodeUnionVol{100}
							\xdef\nodeUnionArea{36}							
							\xdef\nodeContour{\contourNodeThree}		
							
							\xdef\nodeFatherArea{\areaNodeOne}						
							\xdef\nodeFatherVol{\volNodeOne}							
							
							\xdef\nodeChildArea{\areaNodeThree}						
							\xdef\nodeChildVol{\volNodeThree}
								
							\xdef\nodeComputing{3}							
							\xdef\result{-64.22}	
							\xdef\signal{<}					
						}
						{}
					
					\ifthenelse{\i>75 \AND \i<100}
						{
			  				% compute between root and node 1
			  				
							\xdef\nodeUnionVol{212}
							\xdef\nodeUnionArea{60}							
							\xdef\nodeContour{\contourNodeOne}		
							
							\xdef\nodeFatherArea{\areaRoot}						
							\xdef\nodeFatherVol{\volRoot}							
							
							\xdef\nodeChildArea{\areaNodeOne}						
							\xdef\nodeChildVol{\volNodeOne}
								
							\xdef\nodeComputing{1}							
							\xdef\result{-64.93}	
							\xdef\signal{<}					
						}
						{}					
						  	
					% odd and even validation to blanck effect
					
					\ifthenelse{\div=1}
						{									
							
							\ifthenelse{\i<23} % node 2 and 3
								{
									\xdef\blankcolorNodeTwo{purple}
									\xdef\blankcolorNodeThree{blue}										
								}
								{}							
							
							\ifthenelse{\i>22 \AND \i<47} % node 1 and 2
								{
									\xdef\blankcolorNodeOne{purple}									
									\xdef\blankcolorNodeTwo{blue}
								}
								{}
						
							\ifthenelse{\i>46 \AND \i<49} % removing
								{
									\xdef\blankcolorNodeOne{black}					
								}
								{}
							
							\ifthenelse{\i>48 \AND \i<52} % updating
								{
									\xdef\blankcolorNodeOne{green}						
								}
								{}	
														
							\ifthenelse{\i>51 \AND \i<76} % node 1 and 3
								{
									\xdef\blankcolorNodeOne{purple}									
									\xdef\blankcolorNodeThree{blue}							
								}
								{}
								
							\ifthenelse{\i>75 \AND \i<100} % root and node 1
								{
									\xdef\blankcolorRoot{purple}									
									\xdef\blankcolorNodeOne{blue}							
								}
								{}
								
							\xdef\blankcolorFather{purple}
							\xdef\blankcolorChild{blue}
						}
						{
							\xdef\blankcolorRoot{black}						
							\xdef\blankcolorNodeOne{black}	
							\xdef\blankcolorNodeTwo{black}		
							\xdef\blankcolorNodeThree{black}	
							\xdef\blankcolorFather{black}														
							\xdef\blankcolorChild{black}							
						}				
								  															  							  	
			  		% root
			  				
					\node[hasseNode, draw=\blankcolorRoot, root shift=\rootXShift](root){\usebox\treeRootCashTwo};
					
					\node[parameters position=root, text=\blankcolorRoot]{\baselineskip=7pt\vspace{-15pt}
					\begin{align*}\volume(\hat{\tau_0})&=\volRoot \\ |\hat{\tau_0}|&=\areaRoot \\ |\partial \hat{\tau_0}|&=\contourRoot \end{align*} \par};											
							
					\node[label position= root]{$ \tau_0 $,{\color{red}$ \hat{\tau_0} $}};			  								
							
					% node one
											
					\node[hasseNode, draw=\blankcolorNodeOne, root shift=\nOneXShift] (n1) {\usebox\boxNodeOne};
					\node[label position= n1]{$ \tau_1 $,{\color{red}$ \hat{\tau_1} $}};					  	
					  		
					\node[parameters position=n1, text=\blankcolorNodeOne]{\baselineskip=7pt\vspace{-15pt}
					\begin{align*}\volume(\hat{\tau_1})&=\volNodeOne \\ |\hat{\tau_1}|&=\areaNodeOne \\ |\partial \hat{\tau_1}|&=\contourNodeOne \end{align*} \par};
			  			
					% tree label %	
					
					\ifthenelse{\i<100}
						{\node[left = 0.05cm of root]{\scriptsize $ \mathcal{T}_i $};}
						{\node[left = 0.05cm of root]{\scriptsize $ \mathcal{T}^* $};}
						
					% node 2 %			  			
					
				  	\ifthenelse{\i<47}
				  	{
				  							
						\node[hasseNode, draw=\blankcolorNodeTwo, root shift=\nTwoXShift] (n2) {\usebox\treeNodeTwoCashTwo};
					
						\node[parameters position=n2, text=\blankcolorNodeTwo]{\baselineskip=7pt\vspace{-15pt}
						\begin{align*}\volume(\hat{\tau_2})&=12 \\ |\hat{\tau_2}|&=12 \\ |\partial \hat{\tau_2}|&=16 \end{align*} \par};						  																
							
						\node[label position= n2]{$ \tau_2 $,{\color{red}$ \hat{\tau_2} $}};
					
					}{}
						
					% node 3					
				  
					\node[hasseNode, draw=\blankcolorNodeThree, root shift=\nThreeXShift] (n3) {\usebox\treeNodeThreeCashTwo};
				  		
					\node[parameters position=n3, text=\blankcolorNodeThree]{\baselineskip=7pt\vspace{-15pt}
					\begin{align*}\volume(\hat{\tau_3})&=28 \\ |\hat{\tau_3}|&=4 \\ |\partial \hat{\tau_3}|&=8 \end{align*} \par};				  						  						  		

				  	\node[label position= n3]{$ \tau_3 $,{\color{red}$ \hat{\tau_3} $}};
				  		
				  	% draw connections
						
					\ifthenelse{\i<47}
						{ 
							\draw[] (n1) -- (n2)
									(n2) -- (n3); 
									  	
							\draw[green!60, -latex] ($(n2.west)-(0,0.1)$) edge ($(n1.east)-(0,0.1)$)
													($(n3.west)-(0,0.1)$) -- ($(n2.east)-(0,0.1)$);
						}
						{}
															  								
					\ifthenelse{\i>46}
						{
							\draw[] (n1)--(n3); 
										  	
							\draw[green!60, -latex] ($(n3.west)-(0,0.1)$) -- ($(n1.east)-(0,0.1)$);						
						}
						{}						
							
					\draw[] (root)--(n1);
									  				
			  		\draw[green!60, -latex] ($(n1.west)-(0,0.1)$) -- ($(root.east)-(0,0.1)$);
			  			
			  		% equations
			  		
			  		% 7
			  		
			  		\ifthenelse{\i<7 \OR {\i>22 \AND \i<30} \OR {\i>51 \AND \i<59} \OR {\i>75 \AND \i<83}}
			  			{	
			  				\node[equation]{$ \Delta E(\mathcal{T}, \tau_{\nodeComputing})=\dfrac{\nodeUnionVol^2}{\nodeUnionArea}-\textcolor{\blankcolorChild}{\dfrac{\nodeChildVol^2}{\nodeChildArea}}-\textcolor{\blankcolorFather}{\dfrac{\nodeFatherVol^2}{\nodeFatherArea}}+2 \times \nodeContour $};			  				
			  			}
			  			{}			  			
			  	
			  		% 7
			  	
			  		\ifthenelse{\i>6 \AND \i<14 \OR {\i>29 \AND \i<37} \OR {\i>58 \AND \i<66} \OR {\i>82 \AND \i<90}}
			  			{
			  				\node[equation]{$ \Delta E(\mathcal{T}, \tau_{\nodeComputing})=\result$};		  							  				
			  			}
			  			{}
			  			
			  		% 7
			  			
			  		\ifthenelse{\i>13 \AND \i<21 \OR {\i>36 \AND \i<47} \OR {\i>65 \AND \i<73} \OR {\i>89 \AND \i<97}}
			  			{
			  				\node[equation]{$ \Delta E(\mathcal{T}, \tau_{\nodeComputing})=\result$, $ \Delta E(\mathcal{T}, \tau_{\nodeComputing})\signal0 $};
			  			}
			  			{}			  					  	
			  			
			  		% node 2	
			  		
			  		% 3
			  		
			  		\ifthenelse{\i>20 \AND \i<23 \OR {\i>72 \AND \i<76} \OR {\i>96 \AND \i<100}} % not removing
			  			{	
			  				\node[equation]{Não remove $ \tau_{\nodeComputing}$};
			  			}
			  			{}				  						  										  					  			

					\ifthenelse{\i>46 \AND \i<50} % removing
			  			{	
			  				\node[equation]{Removendo $ \tau_2 $};		  							  						  				
			  			}
			  			{}
					
			  		\ifthenelse{\i>49 \AND \i<52} % updating
			  			{				  			
			  				\node[equation]{Atualizando $ \tau_1 $};	  							  						  				
			  			}
			  			{}				

					% end

			  		\ifthenelse{\i>99}
			  			{	
			  				\node[equation]{Fim do exemplo};
			  			}
			  			{}
								
					% removing node two and moving other nodes
					
					\ifthenelse{\i>46 \AND \i<52}			  						  		
						{							
							\pgfmathsetmacro{\temp}{\rootXShift +0.25}
							\xdef\rootXShift{\temp}													
						
							\pgfmathsetmacro{\temp}{\nOneXShift +0.25}
							\xdef\nOneXShift{\temp}
							
							\pgfmathsetmacro{\temp}{\nThreeXShift -0.25}
							\xdef\nThreeXShift{\temp}
						}
						{}

	\end{tikzpicture}
	  \pgfmathtruncatemacro{\temp}{\i+1}%
	  \xdef\i{\temp}%
	  %\end{center}%
	  \newpage%
	}	
\end{document}