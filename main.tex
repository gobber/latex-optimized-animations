% classe base
\documentclass[10pt]{article}

% carregar pacotes necessários 
\usepackage{tikz}
\usepackage{ifthen}
% ...

% extrair os environments como figuras (pdf)
\usepackage[tightpage,active]{preview}

% vamos gerar animações utilizando tikzpicture
\PreviewEnvironment{tikzpicture}

% iniciamos o document
\begin{document}

% geralmente animações tem mais de um frame, assim é necessário utilizar um loop
% \def\Frames{100} = 100 frames ou 100 figuras
\def\Frames{100}

% define um contador para o loop
\xdef\i{0}
\whiledo{\i<\Frames}{%
	\begin{tikzpicture}
		
		% o valor pode ser colocado direto no comando \useasboundingbox
		
		\def\xI{-1} \def\xF{1} \def\yI{-1} \def\yF{1}
		
		% o comando useasboundingbox gera o espaço para criar as animações. Ele não é necessário, porém, sem ele os frames ficam com tamanhos diferentes (dependendo de como são construidos).		
		
		\useasboundingbox (\xI , \yI) rectangle (\xF , \yF);

		% coloque o conteúdo aqui (frame a frame)
	
	\end{tikzpicture}
	% atualiza o contador
	\pgfmathtruncatemacro{\temp}{\i+1}%
	\xdef\i{\temp}%
	% cria uma nova página
	\newpage%
}
\end{document}